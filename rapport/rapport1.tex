\documentclass{article}
\usepackage[french]{babel}
\usepackage[utf8]{inputenc}
\begin{document}
  \title{Stage monitoring audio pour EASEA: \\
    \large Rapport 1}
  \author{PALLAMIDESSI Joseph}
  \maketitle

  \section{Intro} % (fold)
  \label{sec:Intro}
    \paragraph{} % (fold)
    \label{par:} 
      Tout au long de ce stage, je fournirai toutes les deux semaines un rapport de ce
      type, pour que tout les encadrants, même les plus distants puissent suivre la
      progression de ce stage. Cela me permet aussi d'avoir une certaine rigueur dans
      mon travail, surtout leurs de ces phases de recherche, et de garder une trace
      écrite des solutions envisagées et/ou testées.
    % paragraph paragraph name (end)
    
    \paragraph{} % (fold)
    \label{par:}
      Je me suis plus porté sur la faisabilité technique et la recherche des
      technologies à utilisées ces deux premières semaines. J'ai commencé à me
      familiariser avec la plateforme EASEA et j'ai entamé les recherches en
      composition algorithmique. Cela passe aussi par de l'apprentissage de théories
      de la musique plus généraux.
      La production à donc été relativement faible, car il s'agit d'une grande phase de
      recherche initiale.
    
    % paragraph  (end)
  % section Intro (end)

  \section{Buts} % (fold)
  \label{sec:But}
    \paragraph{} % (fold)
    \label{par:}
      Le but de ce stage est de créé un système de monitoring audio pour le solveur
      évolutionnaire distribué EASEA. S'agissant d'un système complexe, les systèmes de
      monitoring usuelles (i.e. graphique) ne se révéle pas être satisfaisant, surtout
      que les runs peuvent être très longues (> 30 heures), et qu'ils force
      l'administrateur/opérateur à fixer son écran... 
    % paragraph  (end)
  % section But (end)

  \section{Problématique} % (fold)
  \label{sec:problematique}
    \paragraph{} % (fold)
    \label{par:}
      La musique doit donc être générée en temps réel, et répondre à certains
      critères essentiels tels qu'une forte cohérence avec l'état du système et le
      fait de ne pas être fatiguante/insupportable sur des écoutes prolongées.
      On songera donc à une certaine cohérence et progression des/de la mélodie(s)
      ainsi qu'à obtenir un résultat relativement harmonieux. 
    % paragraph  (end)
  % section problématique (end)

  \section{Difficultés} % (fold)
  \label{sec:Difficultes}
    Ce projet soulève quelques difficultés notamment sur la génération algorithmique
    de musique tout en gardant une forte association Modèle(EASEA)/Vue(musique
    résultante).
    N'étant pas compositeur ni musicien, j'aurai possiblement besoin de conseil sur
    le plan musical, et je remercie d'avance tout les encadrants pour leurs aides et
    aiguillage sur des domaines qui me sont inconnus.
  % section Difficultés (end)

  \section{Identification des problèmes} % (fold)
  \label{sec:Identification des problemes}
    \paragraph{} % (fold)
    \label{par:}
      Il s'agit dans un premier temps de comprendre qu'est-ce qu'un bon déroulement
      d'un noeud EASEA et du système tout entier, ainsi que de bien identifié les
      problèmes et défaut de fonctionnement, qu'il faudra sonifier différement.
    % paragraph  (end)
    \paragraph{} % (fold)
    \label{par:}
      Por cela je compte me baser sur les outils de monitoring visuelle d'EASEA déjà
      présent, car une grande partie de l'identification des "défauts" de fonctionnement et d'analyse
      statistique à déjà été développé.
    % paragraph  (end)
    \paragraph{} % (fold)
    \label{par:}
      Il aussi que je prenne contacte avec des chercheurs/doctorants travaillant actuellement
      sur EASEA, pour voir quels comportements précis ils veulent surveiller.
    % paragraph  (end)
  % section Identification des "problèmes" (end)

  \section{Mise en place technique} % (fold)
  \label{sec:Mise en place technique}
    \paragraph{} % (fold)
    \label{par:}
      Les technologies utilisées jusqu'ici seront les suivantes:\\
      -pour la synthèse audio, et la gestion du son en règle générale, on utilisera
      le langage de programmation (interprété) SuperCollider, et son interpréteur
      Sclang.\\
      -On utilisera le protocole réseaux OSC, successeur du MIDI, pour communiquer
      au programme écrit en SuperCollider.\\
      \\
      Le système sera tripartite.\\
      Premièrement, une intégration au compilateur EASEA qui intégrera
      au code produit un système de communication au serveur de monitoring audio. Il
      s'agira là de communication TCP/UDP standard en c++ (facile, partie client).\\
      \\
      Un serveur auquel les clients ce connecteront et qui fera une grande partie des
      calcul de composition algorithmique et statistique, écrit en C++
      standard.\\
      \\
      Celui-ci renverra les "notes/plages/ect.." en OSC (Open Sound
      Control) au programme SuperCollider, qui ici fera office de
      "player".\\
      \\
      L'intêret de de gagner du temps (bien meilleurs maîtrise du C++ que du Sc) et
      surtout au niveau des performances (C++ bien plus rapide que l'interpréteur
      SC).
    % paragraph  (end)
  % section Mise en place technique (end)


  \section{Choix du type de composition} % (fold)
  \label{sec:Composition }
    Pour ce garder des possibilités assez (trop?) ouverte, on ne se restreint pas au
    niveau de la liberté compositionnelle: hauteur, tempo, timbre, intensité, etc...
    Et de manière plus abstraite, garder une liberté de "genre".
  % section Composition  (end)

  \section{Canon et contrepoint} % (fold)
  \label{sec:Canon et contrepoint}
    \paragraph{} % (fold)
    \label{par:}
      Il y a des choses intéressantes au niveau de l'écriture en contrepoint et plus
      particulièrement des canons ("Offrandes musicales" de Bach). Domaine relativement
      bien théorisé. Il faut que je me penche plus dessus.
    % paragraph  (end)
  % section Canon et contrepoint (end)

  \section{Sérialisme} % (fold)
  \label{sec:Serielisme}
    \paragraph{} % (fold)
    \label{par:}
    L'approche de la musique sérielle est assez intérressante dans le cadre de ce
    projet, mais je ne sui pas vraiment convaincu du résultat. Peut-être faut-il
    reprendre seulement certains points tel que les transformations de base
    (rétrograge, inversion, etc ...) et en laisser d'autre de coté.   
    % paragraph  (end)
  % section Sérielisme (end)

  \section{Baroque: basse continu et impro blues} % (fold)
  \label{sec:Baroque: basse continu}
    \paragraph{} % (fold)
    \label{par:}
      Je ne me suis pas vraiment penché dessus, mais le principe semble vraiment
      coller avec ce que l'on essaye de faire. \\
      Dans le même principe, les improvisations dans le blues sont aussi facilement
      (en partie) formalisable.
    % paragraph  (end)
  % section section (end)
  \section{Ligeti} % (fold)
  \label{sec:Ligeti}
    \subsubsection{Atmosphère Requiem} % (fold)
    \label{ssub:Atmosphère Requiem}
      \paragraph{} % (fold)
      \label{par:}
    
      % paragraph  (end)
        Il s'agit de pièces pour choeur avec un nombre assez impressionante de voix (dans
        le sens de la composition), avec des clusters assez statique. Les voix du choeur
        suivent des règles qu'on pourrait sans doute formaliser.\\ 
        \\
        Avantage: \\
        Polyphonie complexe régis des règles, on peut tiré parti de la staticité
        apparante. \\
        \\
        Désavantage:\\
        Fatiguant pour l'oreille = > travaille sur les timbre (son plus éthéré)?\\
        "Confus" (trop de voix proche en hauteur) = > étalé sur plus d'octave ? 
    % subsubsection Atmosphère,Requiem (end)
    \subsubsection{Melodien} % (fold)
    \label{ssub:Melodien}
      Ici, on propose de recréée les "textures" présente mesure 0 à 14. On se base
      sur le travail de Marc Chemillier de l'IRCAM.\\
      Présente plus ou moins les mêmes avantages/désavantages que dans le requiem.
      
    % subsubsection subsubsection (end)
  % section Ligeti (end)

  \section{Xenakis} % (fold)
  \label{sec:Xenakis}
    \paragraph{} % (fold)
    \label{par:}
      Utilisation directe de modèle mathématique.
    
    % paragraph  (end)
  % section Xenakis (end)
\end{document}
